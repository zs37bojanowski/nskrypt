\section{VirtualBox}
\subsection{Przygotowanie do pracy SOUbuntu do pracy w VB}
W celu przećwiczenia zadań egzaminacyjnych z przedmiotu pracownia PSO i PSK b edziemy wykorzystywać SO Ubuntu uruchamiany w maszynie wirtualnej VB. Wybór VB zamiast Qemu jest podyktowany znajomością tej maszyny wśród uczniów i łatwością konfiguracji.\newline
\subsection{Instalacja VB na hoście}
Instalacja w Debianie opisana jest np. tutaj: \url{https://vitux.com/how-to-install-virtualbox-on-debian-10}.\newline
\subsection{Podbranie wzorcowego SOUbuntu}
Gotowe do ćwiczenia wzorce (wielu dystybucji Linuxa) można pobrać z \url{https://osboxes.org}. Z obrazów dla VB wybieramy odnośnik do Ubuntu wersji 18. Będzie to wersja językowa British English. Nie jest to dla nas problemem, choć warto pamiętać, że egzamin będzie odbywał się w wersji polskiej. Na ściągnięcie pliku trzeba dużo czasu. Będzie on w formacie *.7z, do którego trzeba zainstalować p7zip. Po rozpakowaniu (\footnote{w razie kłopotów:\url{https://www.simplified.guide/linux/extract-7z-file}} macie gotowy, wstępnie skonfigurowany SOUbuntu z domyślnym użytkownikiem ,,osboxes'' i hasłem ,,osboxes.org''. W tym obrazie SO obowiązuje dla ,,root'' to samo hasło ,,osboxes.org''. Pozostaje obraz ten zaimportować do VB i już można ćwiczyć i ćwiczyć!
