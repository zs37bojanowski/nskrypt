%% -*- coding: utf-8 -*-
\documentclass[a4paper,titlepage,12pt]{mwart}
\setlength{\textheight}{22cm}
\setlength{\textwidth}{18cm}
\setlength{\oddsidemargin}{0.1cm}
\setlength{\evensidemargin}{0.1cm}
\usepackage[english,polish]{babel}
\usepackage[OT1]{fontenc}
\usepackage[utf8]{inputenc}
%\usepackage{iwona}
\usepackage{polski}
\frenchspacing
\usepackage{indentfirst}
\title{\textsc{\huge{NSkrypt PSO/PSK/PLSK firewall ufw}}}
\author{Adam Bojanowski}
\date{5 kwietnia 2020}
%\selectlanguage{english}
\usepackage[polish]{babel}
%\usepackage[OT4]{fontenc}
%\usepackage[utf8]{inputenc}
%\usepackage{iwona}
%\usepackage{polski}
%\usepackage{hyperref}
\usepackage{paralist}
\pagestyle{headings}%{empty}
\usepackage[breaklinks]{hyperref}
\usepackage{graphicx}
\usepackage{moreverb}
\usepackage{enumitem}
\AtBeginDocument{\let\textlabel\label}
\DeclareGraphicsExtensions{.pdf,.png,.jpg}
\begin{document}
\graphicspath{ {/home/navegante/SLES/} }
%\textbf{biblia:}
%\url{http://tldp.org/LDP/intro-linux/html/}\newline
%\url{http://www.tldp.org/LDP/Bash-Beginners-Guide/html/Bash-Beginners-Guide.html}
\subsection{Firewall}
	Ponieważ konfiguracja \textit{iptables}, który jest standardowym firewallem w Linuxie, jest trudna, skorzystamy z programu \textit{ufw}. \texttt{Uncomplicated Firewall} w pełni zasługuje na swą nazwę. Oto na niego przepis:\footnote{\url{http://www.tecmint.com/how-to-install-and-configuufwre-ufw-firewall/}}

	based on https://help.ubuntu.com/community/Gufw
	and on http://guides.webbynode.com/articles/security/ubuntu-ufw.html
	oraz po polsku na: https://webinsider.pl/linux-firewall-ufw/

	Hakerzy mają ułatwione zadanie gdy użytkownicy używają słabych haseł i pozostawiają popularne porty otwarte i na nasłuchu. Wykoanmy kilka kroków dla poprawienia naszego bezpieczeństwa.

\subsubsection{Konfiguracja firewalla ufw\footnote{\url{https://www.tecmint.com/setup-ufw-firewall-on-ubuntu-and-debian/}}}
Ufw jest domyślnie nieaktywny po instalacji i po pierwsze aktywujemy go poleceniem \textit{ufw enable}
Pozostanie aktywny także po restarcie.\newline
Ufw domyślnie nie zamyka żądnych portów. Na początku trzeba zstosować ogólną zasadę bezpieczeństwa polegającą na zamknięciu dostępu do wszystkiego co ie jest nam konieczne:
Domyślna polityka ufw jest opisana w pliku /etc/default/ufw i może być zmieniona:\newline
\textbf{ufw default deny incoming}
\textbf{ufw default allow outgoing}\newline
 pamiętając jednak o pozostawieniu dostępu do serwera za pomocą ssh regułą:
\textbf{ufw allow 22/tcp} lub \textbf{ufw allow ssh/tcp}.\newline

\subsubsubsection{Profile aplikacji ufw}
W /etc/ufw/applications.d znajdują się profile opisujące ustawienia ufw dla aplikacji. Możemy sprawdzić ich listę:\newline
\textbf{ufw app list}\newline
By otrzymać zdefiniowane zasady dla aplikacji, np.:
\textbf{ufw app info 'OpenSSH'} otrzymałem:\newline
\textit{22/tcp}\newline
\subsubsubsection{Definiowanie reguł}
MOżemy definiować reguły wg. nazwy serwisu, numeru portu lub profilu aplikacji, np.: abyotworzyć port 443 https piszemy:\newline
\textbf{ufw allow https} lub\newline
\textbf{ufw allow 443/tcp} lub\newline
\textbf{ufw allow 'Apache Secure'}.\newline
\subsubsubsection{Definiowanie zakresu portów}
Chcąc uruchomić pewne aplikacje w zakresie portów np.:5000-5003 piszemy:
\textbf{ufw allow 5000:5003/tcp}\newline
\textbf{ufw allow 5000:5003/udp}\newline
\subsubsubsection{Definiowanie adresu (adresu i portu)}
Chcąc zezwolić na połączenie na dowolnych portach z konkretnego adresu 192.168.1.66:
\textbf{ufw allow from 192.168.1.66} lub z konkretnego adresu na konkretnym porcie:\newline
\textbf{ufw allow from 192.168.1.66 to any port 22}
\subsubsubsection{Definiowanie podsieci do portu}
Aby zezwolić na łączenie z adresów od 192.168.1.1 do 192.168.1.254 na porcie 22 (ssh):
\textbf{ufw allow from 192.168.1.0/24 to any port 22}
\subsubsubsection{Definiowanie konkretnego interfejsu sieci}
Aby zezwolic na połączenie z konkretnym interfejsem sieci eth2 na konkretnym porcie 22:
\textbf{ufw allow in on eth2 to any port 22}
\subsubsubsection{Blokowanie połączeń}
Przypomnijmy, że domyślnie wszystkie przychodzące połączenia są blokowane za wyjątkiem tych celowo otwartych. Dla przykładu: otworzyłeś porty 80 (http) i 443 (https) i twój webserwer został zaatakowany z sieci 11.12.13.0/24. Naturalnie zablokujesz wszystkie przychodzące ztej sieci połączenia:
\textbf{ufw deny from 11.12.13.0/24} lub blokując tylko otwarte porty:
\textbf{ufw deny from 11.12.13.0/24 to any port 80} 
\textbf{ufw deny from 11.12.13.0/24 to any port 443} 
\subsubsubsection{Usuwanie definicji ufw}
Definicje można usuwać wg. numeru definicji lub wybranej definicji. Aby poznać numer definicji:
\textbf{ufw status numbered)\newline
i by usunąc np. drugą definicję:
\textbf{ufw delete 2}\newline
Aby usunąć wybraną definicję: 
\textbf{ufw delete allow 22/tcp}\newline
\subsubsubsection{Testowanie definicji ufw}
W obawie o skutki działań możemy wypróbować dzięki oopcji \textbf{--dry-run} co się stanie po wykonaniu nowej definicji:
\textbf{ufw delete allow 22/tcp}\newline



\subsubsection{Ogólne zasady konfiguracji firewalla w 5 krokach\footnote{\url{https://www.securitymetrics.com/blog/how-configure-firewall-5-steps}}}.
Jeśli atakujący może uzyskać administracyjny dostęp do firewalla oznacza to koniec gry. Zatem jego ochrona to pierwszy i najważniejszy krok ku bezpieczeństwu. Nigdy nie pracuj zanim nie zabezpieczysz sie poprawnie przez co najmniej:
\subsubsubsection{krok 1}
\begin enumerate
\item {aktualizację;}
\item {usunięcie, zablokowanie, zmianę nazwy każdego domyślnego konta uzytkownika i zminę domyślnych haseł; używaj wyłącznie złożonycj i bezpiecznych haseł;}
\item {jeśli administracja firewallem jest dzielona przez kilku adminów, twórz oddzielne konta administracyjne z prawami ograniczonymi przez obowiązki; nigdy nie używaj współdzelonego konta;}
\item {zablokuj simple network management protocol SNTP lub zabezpiecz go}
\end enumerate
\subsubsubsection{krok 2}
\begin enumerate
\item {opisz chronione obszary i ustal w nich numerację IP bazując na podobnym poziomie wrażliwości i funkcjonalności;}
\end enumerate
\subsubsubsection{krok 3}
\begiin enumerate
\item {Skonfiguruj dostęp ACL}
Ruch jest określony za pomocą zasad (ACL) stosowanych dokażdego interfejsu firewalla. Określ adresy IP źródłowe i docelowe i adresy portów gdy to możliwe. Na końcu każdej ACL-użyj zawsze ..deny all'' by odfiltrować zbędny ruch. Zaleca się unimożliwić zarządznie zarówno po ssh jak i web ze świata. Koniecznie wyłącz każdy nieszyfrowany protokół dla zarządzania np. telnet i http.
\end enumerate
\subsubsubsection{krok 4}
\begin enumerate
\item {testowanie}
Powinno skłądać się z testów zagrożeń i penetracyjnych
\end enumerate
\subsubsubsection{krok 4}
\begin enumerate
\item {zarządzanie}
Aktualizacja, przeglądanie logów, skanowanie zagrożeń, ocena zasad ACL firewalla co min. pół roku.
\end enumerate
KOniecznie dokumentuj każdy swój krok.

\end{document}
